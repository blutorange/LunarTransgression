%% A simple template for a lab or course report using the Hagenberg setup
%% with the standard LaTeX 'report' class

%\documentclass[english,11pt]{report}		
\documentclass[ngerman,11pt]{report}



\usepackage[a4paper, total={6in, 8in}]{geometry}
\usepackage{hgb}
\usepackage{hgbbib}
\usepackage{hgbheadings}
\usepackage{tabularx}

\graphicspath{{images/}}   % where are the images?
\bibliography{literatur}  % requires literatur.bib 

\author{Andre Wachsmuth}
\title{Programmierprojekt des Moduls Verteilte Systeme \\ Client-Server-Multiplayer-Spiel}
\date{August 2017}

%%%----------------------------------------------------------
\begin{document}
%%%----------------------------------------------------------
\maketitle
\tableofcontents
%%%----------------------------------------------------------

\chapter*{Summary}

\begin{quote}
 \verb!\documentclass[english]{hgbtermreport}! 
\end{quote}
at the top of this document by
\begin{quote}
 \verb!\documentclass[german]{hgbtermreport}!.
\end{quote}
Also, you may want to place the text of the individual chapters in separate files and 
include them using \verb!\include{..}! 
(see file \verb!_DaBa.tex!).

Simply omit this chapter (which could also be called ``Abstract'') if you do not want to provide this kind of summary.


%%%----------------------------------------------------------
\chapter{Websocket-Kommunikation}
%%%----------------------------------------------------------

Die Kommunikation zwischen Clients und dem Server erfolgt über Websockets. Websocket
sind ein auf \textsc{tcp} basierendes zustandbehaftetes Plain-Text-Protokol, die 
ine bidirektionale Kommunikation für Webanwendungen ermöglicht.

Session-Verwaltung und die Persistierung von Daten während der Session wird von Websockets bereitgestellt.
Der Datenaustausch geschieht mittels Textnachrichten. Zu lösende Probleme sind somit die Authorisierung
von Sessions und die Struktur (Protokoll) der zwischen Client und Server ausgetauschten Nachrichten.

Für die Struktur der Nachrichten wird das \textsc{json}-Format verwendet. Es wird von vielen
Software-Implementierungen unterstüzt. Die zusätzliche Ausdrucksfähigkeit und damit verbundene
Komplexität von \textsc{xml} wird nicht benötigt. Im folgenden wird die \textsc{json}-Struktur
beschrieben. Jede Nachricht beginnt mit der Metanachricht \textsc{lunarmessage}, die folgende
Informationen enthält, siehe hierzu auch Tabelle \ref{table:lunarmessage}.

Der Eintrag status gibt den Verarbeitungsstatus an, dieser ist \textsc{ok} bei korrekter
Verarbeitung. Mögliche Status sind in der Tabelle \ref{table:messagesStatus} zusammengestellt.

Im Eintrag payload werden nachrichtenspezifische Informatione mitgesendet und der Eintrag
type gibt an, um welchen Typ es sich handelt, siehe \ref{table:messageType}. Die Zeichenkette
im Eintrag payload ist ein \textsc{json}-String.

\textsc{tcp} garantiert die Reihenfolge des Nachrichtenempfangs in Reihenfolge des Versandes,
aber nicht die logische Reihenfolge und die Verarbeitung in dieser Reihenfolge durch die
Anwendung. Der Eintrag time dient der Ordnung der Nachrichten. Der Sender setzt bei Beginn
der Websocket-Session einen Laufindex auf null und sendet diesen im Eintrag time bei jeder
Nachricht mit an den Empfänger. Dabei wird der Laufindex bei jeder Nachricht um eins erhöht.
Der Empfänger \textsc{muss} die empfangenen Nachrichten in dieser Reihenfolge bearbeiten.

Jede empfangene Nachricht wird mit einer \textsc{lunarmessage} vom Typ \textsc{received} quittiert,
die den Laufindex nicht erhöht und im payload den Laufindex der quittierten Nachricht enthält.
Diese Nachricht dient beispielweise dazu, dass der Sender die Nachricht aus dem eventuell vorhandenen
implementierungspezifischen Cache löschen kann.
\footnote{Dieser Mechanismus ist derzeit noch nicht implementiert.}

Sind bereits weitere Nachrichten mit höheren Laufindex beim Empfänger eingetroffen und ist ein gewisses
Zeitlimit überschirtten, \textsc{kann} der Empfänger durch Senden einer Metanachricht mit dem
type \textsc{repeat} die Nachricht erneut anfordern. Im Eintrag payload ist der Laufindex der
angeforderten Nachricht enthalten. Der Nachrichtentyp \textsc{repeat} erhöht den Laufindex time nicht.
\footnote{Dieser Mechanismus ist derzeit noch nicht implementiert.}

Der Server reagiert nur auf Nachrichten beziehungsweise informiert den Client über Ereignisse,
erwartet aber keine Antworten. Clients senden immer Nachrichten in Erwartung einer Antwort.
Zur Reduzierung des Programmieraufwandes der Client-Software wurde daher ein
Request-Response-Mechanismus für Nachrichten des Clients implementiert. Alle Nachrichten, deren
type auf \textsc{response} endet, enthalten im payload-\textsc{json} den Eintrag origin, der
den Laufindex der Nachricht enthält, auf die geantwortet wird.

Clients werden bereits vor dem Aufbau der Websocket-Verbindung mittels einer Login-Eingabemaske
auf der Webseite authorisiert. Dabei wird ein zeitbegrenztes Token erstellt und dem Client
übermittelt. Initial wird beim Öffnen einer Websocket-Session diese als unauthorisiert markiert
und bis auf \textsc{authorize} keine anderen Nachrichten akzeptiert. Das Token und der Nutzername wird in
der \textsc{authorize}-Nachricht an den Server übermittelt, der mit \textsc{authorize\char`_response} antwortet.
Bei erfolgreicher Authorisierung wird die Session als authorisiert markiert und der Statuscode \textsc{ok}
übermittelt, \textsc{generic error} andernfalls.

\begin{center}
\textbf{Don't just show your program code!} 
\end{center}

By the way, all you ever need to know about image processing (and more) can be 
found in \cite{BurgerBurge08}.%

%%%----------------------------------------------------------
\chapter{Server}
%%%----------------------------------------------------------

\section{Eingesetzte Technologien}

Auf dem Server wird \textsc{jsf} (Java Server Faces) eingesetzt, konkreter
Primefaces\footnote{https://www.primefaces.org/ Eine Übersicht über alle
Funktionionen ist unter https://www.primefaces.org/showcase/ zu finden.}.
Das System kann auf verschiedenen Servlet-Containern laufen, zur Entwicklung
wurde Tomcat 8.5 genutzt.

Um periodische Aufgaben auszuführen, wird der
Quartz-Scheduler\footnote{http://www.quartz-scheduler.org/} eingesetzt.

Die Persistierung von Daten nutzt eine Datenbank. Aus Zeitgründen wird ein ORM
(Object Relational Mapping) genutzt, konkret \textsc{jpa} (Java Persistence API)
mit Hibernate\footnote{http://hibernate.org/} als JPA-Provider. Weiterhin wird
HikariCP\footnote{https://github.com/brettwooldridge/HikariCP} eingesetzt, um
Verbindungen zur Datenbank zu verwalten. HikariCP stellt einen Pool an
aufrechgehaltenen Datenbankverbindungen bereit, sodass nicht für jede Anfrage
eine Verbindung geöffnet werden muss.

Das \textsc{erm} für das \textsc{orm} ist in der Abbildungen zu finden. Die Hauptentität
ist die Entität \textsc{player}, welche Assoziatonen auf die \textsc{item}s und
\textsc{characterstate}s des Spielers hat. Diese Informationen werden bei Beginn der
Session aus der Datenbank geladen und zwischengespeichert, um unnötige Datenbankanfragen
zu vermeiden.

\textsc{item}, \textsc{character}, \textsc{item} sind immutable Entitäten, diese Daten
werden einmalig bei der Installation in die Datenbank eingespielt. \textsc{player} und
\textsc{characterstate} sind mutabel und werden öfters im Laufe des Programms geändert.

Der Server ist für die Logik und die Konsistenz der Daten verantwortlich und muss davon
ausgehen, dass es böswillige Client geben kann, demgemäß müssen alle Eingaben und Nachrichten
validiert werden. Dies bedeutet auch, dass die Client-Software nicht für Datenkonsistenz
verantworlich ist, was den Programmiertaufwand reduziert.

Schließlich seien noch kurz weitere eingesetzte Bibliotheken erwähnt: Google
Dagger\footnote{https://github.com/google/dagger} für Dependency Injection,
Jsoniter\footnote{http://jsoniter.com/} für die Serialisierung und
Deserialisierung von Objekten, slf4j für Logging.

\section{Servlets}

Die eingesetzten clientseitigen Technologien erfordern, dass mediale Ressourcen mittels
als \textsc{http}-Response ausgeliefert werden. Es gibt statische und dynamische Inhalte.
Erstere können direkt über den Servlet-Container bereitgestellt werden. Dynamische Inhalte
wie beispielsweise Grafiken der Charaktere, über die der Spieler momentan verfügt, können
nicht statisch ausgeliefert werden. Hierzu gibt es zwei \textsc{http}-Servlets.

Das Resourcen-Servlet stellt Dateien bereit, die dynamisch in der Datenbank gespeichert sind.
Die Dateien werden nicht durch den Nutzer bereitgestellt und liegen somit in der Kontrolle des
Programmierers beziehungsweise des Administrators. Zudem ist keine hierarchische Gliederung
der Dateien notwendig. Daher wurde ein einfaches Datenbankschema mit einem eindeutigen Namen
als primärer Schlüssel gewählt. Es muss darauf geachtet werden, keine doppelten Namen zu nutzen.

Das Spritesheet-Servlet erlaubt das Abholen mehrerer Dateien mit einem \textsc{http}-Request.
Ziel ist es, zu vermeiden, dass clientseitig bis zu fünfzig und mehr kleinere Bilddateien im
KiB-Bereich angefordert werden müssen. Dazu wird serverseitig eine große Bilddatei generiert,
die die kleineren Bilder enthält und serverseitig in einem Cache gehalten.
In einer dazugehörigen \textsc{json}-Datei ist festgehalten, an welcher Position sich jedes Bild befindet. Der Client fordert zuerst das \textsc{json} an und anschließend die Bilddatei. Der Cache
dient dazu, bei der zweiten Anforderung die Bilddatein nicht erneut generieren zu müssen und die
Einträge haben daher eine kurze Lebensdauer.

%%%----------------------------------------------------------     
\chapter{Client}
%%%----------------------------------------------------------

Die Clientsofware ist ES6-JavaScript und läuft in einem Webbrowser. Die Aufgabe
des Clients besteht in der grafischen Darstellung des Programmzustands zur
Nutzerinteraktion und Weiterleitung der Nutzereingaben an den Server.

Für die grafische Darstellung wird die Bibliothek PixiJS\footnote{http://www.pixijs.com/}
verwendet, die \glqq The HTML5 Creation Engine\grqq. PixiJS nutzt \textsc{WebGL} mit
Fallback auf Canvas, wenn \textsc{WebGL} nicht zur Verfügung steht. Diese Bibliothek
wurde gewählt, da sie aktuell und lightweight ist sowie  eine gute Dokumenation mit
Beispielen besitzt.

Für die Ausgabe von Audio wird die Bibliothek HowlerJS\footnote{https://howlerjs.com/}
verwendet. Diese Bibliothek unterstützt alle gängigen Formate mit Fallbacks. In
diesem Projekt wird das Format webm verwendet, da es allen modernen Browser Stunden
unterstützt wird.

%%%----------------------------------------------------------     
\chapter{Einrichtung und Installation}
%%%----------------------------------------------------------

More chapters will follow.


This could be a good place to describe how you experienced this course, what you liked or didn't, and to provide suggestions for improvement.

%%%----------------------------------------------------------     
\chapter{Abbildungen und Tabellen}
%%%----------------------------------------------------------

\begin{table}[]
\centering
\caption{Einträge der \textsc{json}-Struktur der Metanachricht \textsc{lunarmessage}}
\label{table:lunarmessage}
\begin{tabularx}{\textwidth}{l|l|X}
Eintrag & Datentyp & Erläuterung \\ \hline
time    & int    & Eine bei null beginnende Ganzzahl. Dient der Synchronisation der Nachrichten. \\
type    & string & Art der Nachricht, gibt an, wie der Eintrag payload interpretiert wird  \\
status  & int    & Verarbeitungsstatus, beispielweise 0 für \textsc{ok}                          \\
payload & string & Typenspezifische Nachricht. Enthält einen \textsc{json}-String.       
\end{tabularx}
\end{table}


\begin{table}[]
\centering
\caption{Mögliche Verabeitungsstatus in der Metanachricht \textsc{lunarmessage}}
\label{table:messagesStatus}
\begin{tabularx}{\textwidth}{l|l|X}
Bezeichnung            & Wert & Erläuterung \\ \hline
\textsc{ok}            & 0    & Fehlerfreie Verarbeitung. \\
\textsc{generic error} & 20   & Nicht näher spezifizierter Fehler in der Verarbeitung.  \\
\textsc{access denied} & 21   & Fehlende Authorisierung. Zur Authorisierung muss die
                                Nachricht \textsc{authorize} gesendet werden. \\
\textsc{timeout}       & 22   & Überschreitung des Zeitlimits bei der Verarbeitung.
\end{tabularx}
\end{table}


\begin{table}[]
\centering
\caption{Nachrichtentypen im Eintrag payload einer \textsc{lunarmessage}}
\label{table:messageType}
\begin{tabularx}{\textwidth}{l|X}
Bezeichnung & Erläuterung \\ \hline
\textsc{authorize}                           & Authorisierung der Session \\
\textsc{authorize\char`_response}            & Status \textsc{ok}, wenn authorisiert\\
\textsc{fetch\char`_data}                    & Ahbolen von benötigten Daten, beispielweise Daten von Items\\
\textsc{fetch\char`_data\char`_response}     & Antwort mit denangeforderten Daten. \\
\textsc{invite}                              & Einladen eines anderen Spielers \\
\textsc{invite\char`_response}               & Antwort. \\
\textsc{invited}                             & Nachricht vom Server an Client, wenn Einladung von anderem Client vorliegt \\
\textsc{invite\char`_accept}                 & Client lädt einen anderen Client ein \\
\textsc{invite\char`_accept\char`_response}  & Antwort. \\
\textsc{invite\char`_accepted}               & Nachricht vom Server an Client, wenn anderer Client Einladung akzeptiert hat \\
\textsc{invite\char`_retract}                & Client zieht Einladung zurück \\
\textsc{invite\char`_retract\char`_response} & Antwort. \\
\textsc{invite\char`_retracted}              & Nachricht vom Server an Client, wenn anderer Client die Einladung zurückgezogen hat \\
\textsc{invite\char`_reject}                 & Client akzeptiert Einladung nicht \\
\textsc{invite\char`_reject\char`_response}  & Antwort. \\
\textsc{invite\char`_rejected}               & Nachricht vom Server an Client, wenn anderer Client Einladung nicht akzeptiert hat \\
\textsc{prepare\char`_battle}                & Client sendet Kampfkonfiguration \\
\textsc{prepare\char`_battle\char`_response} & Antwort. \\
\textsc{step\char`_battle}                   & Client sendet Kampfzugkonfiguration \\
\textsc{step\char`_battle\char`_response}    & Antwort. \\
\textsc{battle\char`_prepared}               & Nachricht vom Server an beide Clients, wenn Kampfvorbereitung abgeschlossen \\
\textsc{battle\char`_stepped}                & Nachricht vom Server an beide Clients, wenn Kampfzugberechnung abgeschlossen \\
\textsc{battle\char`_cancelled}              & Nachricht vom Server an beide Clients, wenn Kampf abgebrochen wurde \\
\textsc{battle\char`_ended}                  & Nachricht vom Server an beide Clients, wenn Kampfende erreicht wurde \\
\textsc{loot}                                & Client sendet Beuteinformationen \\
\textsc{loot\char`_response}                 & Antwort. \\
\textsc{unknown}                             & Fallback für unbekannte Nachrichten
\end{tabularx}
\end{table}


%%%----------------------------------------------------------
\MakeBibliography
%%%----------------------------------------------------------

\end{document}

